\chapter{Evaluation}\label{chap:evaluation}

\section*{}

This project was made with the support and constant review of the collaborating team at Synopsys, adapting every iteration towards the final goal product. For a final validation process, it was asked the IPK R\&D team leader and manager short considerations about the whole project and process itself.

\section{Setup}

A prototyping kit is quite a complex design in great extent due to its nature of multiple configurations. To automate the process of test compliance of the prototyping kits and with it reducing considerably the effort needed during the testing phase, jobs were implemented in Jenkins to automate the testing of the kits during their compliance testing phase. Nonetheless, the configurations to be tested scale quite exponentially and maintain their traceability can be a time consuming effort. In addition, the life time of the IPK with new versions occurring at a very fast pace makes the analysis of data extremely cumbersome, erroneous and time consuming.


\section{Requirements}

The requirements for the development of the Jenkins’ Dashboard were to improve the analysis, readability, usability and traceability during the lifecycle span of the IPKs. The Dashboard should arrange and consolidate the most need information related with the parameters of a specific project build, this way the product development can be monitored in a single snapshot. 

\section{Results}

The Dashboard developed appears to be a powerful tool to help reducing the analysis of the state of specific IPK configuration and respective version. It fills the requirements of filtering and keep traceability of the specific relevant information of each configuration. In addition, it is very effective as it is accurate and complete in showing the defined important metadata information of the specific product configuration. Accurate, because it shows the job state of the builds to the defined information (metadata), namely product version or configuration of the image tested. Complete, since the data is well presented. The efficiency is also improved as the time required to find the successful and unsuccessful builds is now reduced. Moreover, it is easier to categorize which are the IPK product configurations that require less effort to successful pass the compliance test during the testing phase. 

\section{Conclusions}

The overall Dashboard satisfies greatly all the requirements proposed, some works need to be done to improve the cosmetic aspect of the Dashboard to make it more appealing to the eye. It is now a work of the IPK team to update the jobs so that the Dashboard can be used at its full extent. 
