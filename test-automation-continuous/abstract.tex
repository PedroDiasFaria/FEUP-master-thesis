\chapter*{Resumo}

Assim como no desenvolvimento de software onde os bugs devem ser detectados e corrigidos o mais rapidamente possível, a validação de hardware está ao mesmo nível de prioridade quando usada em aplicações que procuram reduzir drasticamente os custos e estender a vida útil dos sistemas.

\sloppy
Nestes sistemas embutidos complexos, vários sistemas de interface, tais como os Field-Programmable Gate Arrays (FPGAs), são distribuídos em vários blocos de hardware.

A interação entre os vários sistemas envolve uma integração do firmware e o seu hardware.
Com a complexidade crescente desses sistemas, especialmente nessa interação, é imperativo reduzir o tempo do processo de validação funcional do hardware.

Para se alcançar isso, propomos implementar um ambiente de hardware de validação automática, permitindo testes sistemáticos e automáticos no hardware, assim como mostrar métricas de desempenho do hardware com uma combinação específica de cada elemento na placa do sistema, como a versão do Driver no CPU, a versão do Verilog configurada no FPGA, assim como a própria versão da placa.

Nesta tese, realizada na Synopsys Porto, será descrita a estrutura de um ambiente de validação de testes automáticos em integração contínua, e aplicá-lo no contexto de validação de hardware. 

Será também delineado o processo de criação um painel com as diferentes informações relacionadas aos sistemas em teste, com o objetivo de ajudar os stakeholders numa equipa de desenvolvimento a ter uma vista dos resultados do processo mais simplificada.

\vspace*{10mm}\noindent

\noindent\textbf{Palavras Chave}: \emph{Continuous Integration}, \emph{Hardware Validation}

\vspace*{5mm}\noindent

\noindent\textbf{Classificação}: 
\begin{itemize}
\item \emph{Software and its engineering $\rightarrow$ Software creation and management $\rightarrow$ Software verification and validation}; 
\item \emph{Hardware $\rightarrow$ Hardware validation}
\end{itemize}